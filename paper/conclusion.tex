\section{Conclusion}

This study provides a comprehensive empirical comparison between single-neuron and dual-neuron output layer configurations for binary classification neural networks. Through extensive experimentation with multiple architectures and datasets, we have identified consistent patterns that provide valuable insights for both theoretical understanding and practical implementation.

\subsection{Summary of Key Findings}

Our research revealed several important findings:

\begin{enumerate}
\item \textbf{Performance Advantage}: The single-neuron sigmoid approach consistently outperformed the dual-neuron softmax approach across all tested architectures and datasets, though the magnitude of this advantage varied.

\item \textbf{Architecture Dependence}: The performance gap between approaches was more pronounced in smaller architectures (e.g., our custom CNN) compared to larger pre-trained models (ResNet50), suggesting that output layer design becomes less critical as model capacity increases.

\item \textbf{Training Dynamics}: Single-neuron models typically converged faster and exhibited more stable learning curves, with fewer fluctuations in validation loss during training.

\item \textbf{Generalization Capability}: The single-neuron approach demonstrated better generalization to test data, with smaller gaps between training and test performance, particularly in smaller network architectures.

\item \textbf{Parameter Efficiency}: Beyond the obvious parameter reduction (one output neuron instead of two), the single-neuron approach appears to make more efficient use of model capacity, especially in constrained model architectures.
\end{enumerate}

\subsection{Contributions to the Field}

This work makes several contributions to the understanding of neural network design for binary classification:

\begin{enumerate}
\item It provides the first systematic, empirical comparison of output layer architectures specifically for binary classification across different network backbones.

\item It establishes a clear set of practical guidelines for practitioners implementing binary classification systems, potentially improving model performance in real-world applications.

\item It offers insights into the theoretical aspects of model optimization and decision boundary formation in neural networks, connecting empirical results to theoretical frameworks.
\end{enumerate}

\subsection{Future Research Directions}

Based on our findings, several promising directions for future research emerge:

\begin{enumerate}
\item \textbf{Extension to Multi-Label Problems}: Investigating whether similar patterns hold for multi-label classification, where multiple binary decisions are made simultaneously.

\item \textbf{Probability Calibration Analysis}: A deeper exploration of probability calibration properties of both approaches, particularly in risk-sensitive applications.

\item \textbf{Architecture Search}: Developing automated methods to determine the optimal output layer configuration based on task characteristics and model architecture.

\item \textbf{Theoretical Analysis}: Formal mathematical analysis of optimization dynamics in both approaches to provide stronger theoretical foundations for the observed empirical differences.

\item \textbf{Domain Adaptation}: Examining how the choice of output layer influences transfer learning and domain adaptation capabilities in binary classification scenarios.
\end{enumerate}

In conclusion, while both single-neuron and dual-neuron approaches are viable for binary classification, our research provides strong evidence favoring the single-neuron sigmoid approach in most practical scenarios. This seemingly minor architectural choice can yield meaningful improvements in model performance, especially in smaller networks or data-constrained settings.
