\subsubsection{Learning Stability}

We observed differences in the stability of the learning process between the two approaches:

\begin{enumerate}
\item \textbf{Validation Loss Fluctuations}: The dual-neuron models tended to exhibit greater fluctuations in validation loss during training. This was particularly evident in the Small CNN experiments, where the validation loss curve for the dual-neuron model showed more pronounced oscillations.

\item \textbf{Overfitting Tendency}: The dual-neuron approach showed slightly higher susceptibility to overfitting, especially in the Small CNN architecture. This manifested as a growing gap between training and validation accuracy as training progressed.

\item \textbf{Learning Rate Sensitivity}: Both approaches benefited from learning rate scheduling, but the dual-neuron models appeared more sensitive to learning rate adjustments, often showing more dramatic improvements after learning rate reductions.
\end{enumerate}

These observations suggest that the single-neuron approach may offer a more stable optimization path, potentially due to its more constrained parameter space.

\subsection{Generalization Capability}

A critical aspect of any machine learning model is its ability to generalize beyond the training data. Our experiments revealed several important insights regarding the generalization capabilities of single-neuron versus dual-neuron approaches.

\subsubsection{Test Performance}

When evaluating on held-out test data, we observed that both approaches generalized well, but with consistent differences:

\begin{enumerate}
\item \textbf{Generalization Gap}: The single-neuron models maintained their performance advantage on test data. For the Small CNN on airplane vs. automobile classification, the test accuracy was 0.9835 for the single-neuron approach compared to 0.9735 for the dual-neuron approach.

\item \textbf{Robustness Across Class Pairs}: The generalization advantage of the single-neuron approach was consistent across different binary classification tasks, including more challenging class pairs with higher visual similarity (like cat vs. dog).

\item \textbf{Performance on Edge Cases}: Qualitative analysis of misclassifications showed that both approaches struggled with similar edge cases, but the dual-neuron approach typically had a higher error rate on these challenging examples.
\end{enumerate}

\subsubsection{Training-Test Performance Gap}

The gap between training and test performance provides insights into potential overfitting:

\begin{enumerate}
\item \textbf{Small CNN Models}: The dual-neuron approach showed a larger gap between training and test accuracy (approximately 2-3\% difference) compared to the single-neuron approach (approximately 1-2\% difference), suggesting slightly higher overfitting tendencies.

\item \textbf{ResNet50 Models}: Both approaches maintained similar training-test gaps with ResNet50, likely due to the regularizing effect of transfer learning from pre-trained weights.
\end{enumerate}

These observations suggest that the single-neuron approach may offer better regularization properties, particularly in smaller network architectures. This could be attributed to the more constrained parameter space of the single-neuron output layer, which may help prevent the model from fitting noise in the training data.

\subsection{Architecture-Specific Effects}

Our experiments with different neural network architectures revealed interesting interactions between the network backbone and the output layer configuration.

\subsubsection{Small CNN vs. ResNet50}

Comparing the performance differences across architectures:

\begin{enumerate}
\item \textbf{Magnitude of Performance Gap}: The performance gap between single-neuron and dual-neuron approaches was more pronounced in the Small CNN architecture (accuracy difference of 0.0100) compared to the ResNet50 architecture (accuracy difference of 0.0030 for frog vs. ship classification).

\item \textbf{Overall Performance Ceiling}: ResNet50 models achieved higher absolute performance regardless of output layer configuration, with both approaches reaching $>$99\% accuracy on some class pairs. This suggests that for sufficiently powerful models, the choice of output layer may have diminished importance.

\item \textbf{Parameter Efficiency}: The relative parameter efficiency of the single-neuron approach is more significant in smaller models like our custom CNN, where the output layer represents a higher proportion of total parameters.
\end{enumerate}

\subsubsection{Transfer Learning Effects}

For models using transfer learning (ResNet50):

\begin{enumerate}
\item \textbf{Feature Extraction Quality}: Both output layer approaches benefited similarly from the high-quality features extracted by pre-trained ResNet50 layers.

\item \textbf{Fine-Tuning Dynamics}: During fine-tuning, the single-neuron models required slightly less adaptation of the pre-trained features, suggesting better compatibility with general visual features extracted by ImageNet-trained networks.

\item \textbf{Convergence with Limited Data}: When training with reduced dataset sizes, the single-neuron approach showed more robust performance, particularly with the ResNet50 architecture, suggesting better generalization from limited examples.
\end{enumerate}

These findings suggest that while the single-neuron approach consistently outperforms the dual-neuron approach, the magnitude of this advantage varies with network architecture. The performance gap appears to narrow as model capacity increases, though the single-neuron approach maintains its edge even in high-capacity models.
